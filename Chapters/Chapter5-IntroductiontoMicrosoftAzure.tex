% Chapter 5
% !TEX root = ../main.tex
\chapter{Introduction to Microsoft Azure Cloud} % Main chapter title

\label{Chapter5} % For referencing the chapter elsewhere, use \ref{Chapter1}

%----------------------------------------------------------------------------------------

% Define some commands to keep the formatting separated from the content
% \newcommand{\keyword}[1]{\textbf{#1}}
% \newcommand{\tabhead}[1]{\textbf{#1}}
% \newcommand{\code}[1]{\texttt{#1}}
% \newcommand{\file}[1]{\texttt{\bfseries#1}}
% \newcommand{\option}[1]{\texttt{\itshape#1}}
%----------------------------------------------------------------------------------------

\section{Connected Services in Xamarin Studio}
Welcome to this \LaTeX{} Thesis Template, a beautiful and easy to use template for writing a thesis using the \LaTeX{} typesetting system.
%https://azure.microsoft.com/en-us/features/xamarin/ refer this link its good one
If you are writing a thesis (or will be in the future) and its subject is technical or mathematical (though it doesn't have to be), then creating it in \LaTeX{} is highly recommended as a way to make sure you can just get down to the essential writing without having to worry over formatting or wasting time arguing with your word processor.

\LaTeX{} is easily able to professionally typeset documents that run to hundreds or thousands of pages long. With simple mark-up commands, it automatically sets out the table of contents, margins, page headers and footers and keeps the formatting consistent and beautiful. One of its main strengths is the way it can easily typeset mathematics, even \emph{heavy} mathematics. Even if those equations are the most horribly twisted and most difficult mathematical problems that can only be solved on a super-computer, you can at least count on \LaTeX{} to make them look stunning.

%----------------------------------------------------------------------------------------

\section{Azure App Services}

App Service is a platform-as-a-service (PaaS) offering of Microsoft Azure. Create web and mobile apps for any platform or device. Integrate your apps with SaaS solutions, connect with on-premises applications, and automate your business processes. Azure runs your apps on fully managed virtual machines (VMs), with your choice of shared VM resources or dedicated VMs.

App Service includes the web and mobile capabilities that we previously delivered separately as Azure Websites and Azure Mobile Services. It also includes new capabilities for automating business processes and hosting cloud APIs. As a single integrated service, App Service lets you compose various components -- websites, mobile app back ends, RESTful APIs, and business processes -- into a single solution.


App types in App Service

App Service offers several app types, each of which is intended to host a specific workload:
Web Apps - For hosting websites and web applications.
Mobile Apps For hosting mobile app back ends.
API Apps - For hosting RESTful APIs.
Logic Apps - For automating business processes and integrating systems and data across clouds without writing code.



\subsection{Web Apps}

App Service Web Apps is a fully managed compute platform that is optimized for hosting websites and web applications. This platform-as-a-service (PaaS) offering of Microsoft Azure lets you focus on your business logic while Azure takes care of the infrastructure to run and scale your apps.


\subsection{Mobile Apps}

Azure App Service is a fully managed Platform as a Service (PaaS) offering for professional developers that brings a rich set of capabilities to web, mobile and integration scenarios. Mobile Apps in Azure App Service offer a highly scalable, globally available mobile application development platform for Enterprise Developers and System Integrators that brings a rich set of capabilities to mobile developers.


Mobile App Features

The following features are important to cloud-enabled mobile development:

Authentication and Authorization - Select from an ever-growing list of identity providers, including Azure Active Directory for enterprise authentication, plus social providers like Facebook, Google, Twitter and Microsoft Account. Azure Mobile Apps provides an OAuth 2.0 service for each provider. You can also integrate the SDK for the identity provider for provider specific functionality.
Discover more about our authentication features.
Data Access - Azure Mobile Apps provides a mobile-friendly OData v3 data source linked to SQL Azure or an on-premises SQL Server. This service can be based on Entity Framework, allowing you to easily integrate with other NoSQL and SQL data providers, including Azure Table Storage, MongoDB, DocumentDB and SaaS API providers like Office 365 and Salesforce.com.
Offline Sync - Our Client SDKs make it easy for you to build robust and responsive mobile applications that operate with an offline data set that can be automatically synchronized with the backend data, including conflict resolution support.
Discover more about our data features.
Push Notifications - Our Client SDKS seamlessly integrate with the registration capabilities of Azure Notification Hubs, allowing you to send push notifications to millions of users simultaneously.
Discover more about our push notification features.
Client SDKs - We provide a complete set of Client SDKs that cover native development (iOS, Android and Windows), cross-platform development (Xamarin for iOS and Android, Xamarin Forms) and hybrid application development (Apache Cordova). Each client SDK is available with an MIT license and is open-source.




\subsection{API Apps }
There are a multitude of mathematical symbols available for \LaTeX{} and it would take a great effort to learn the commands for them all. The most common ones you are likely to use are shown on this page:
\url{http://www.sunilpatel.co.uk/latex-type/latex-math-symbols/}

You can use this page as a reference or crib sheet, the symbols are rendered as large, high quality images so you can quickly find the \LaTeX{} command for the symbol you need.

\subsection{Logic Apps  }

The \LaTeX{} distribution is available for many systems including Windows, Linux and Mac OS X. The package for OS X is called MacTeX and it contains all the applications you need -- bundled together and pre-customized -- for a fully working \LaTeX{} environment and work flow.

MacTeX includes a custom dedicated \LaTeX{} editor called TeXShop for writing your `\file{.tex}' files and BibDesk: a program to manage your references and create your bibliography section just as easily as managing songs and creating playlists in iTunes.
