% Chapter 3 % Not completed on going still

\chapter{Mobile Application Development} % Main chapter title

\label{Chapter3} % For referencing the chapter elsewhere, use \ref{Chapter1}

%----------------------------------------------------------------------------------------

% Define some commands to keep the formatting separated from the content
% \newcommand{\keyword}[1]{\textbf{#1}}
% \newcommand{\tabhead}[1]{\textbf{#1}}
% \newcommand{\code}[1]{\texttt{#1}}
% \newcommand{\file}[1]{\texttt{\bfseries#1}}
% \newcommand{\option}[1]{\texttt{\itshape#1}}


%----------------------------------------------------------------------------------------

\section{Mobile Platforms}

According to a market share study by IDC \parencite{Reference1}, the smartphone market is
currently clearly dominated by Android, which held over 86.8 percent of the
market during the second quarter of 2016. Meanwhile, iOS saw its market share for 2016Q3 grow by 12.7 percent QoQ with 45.5 million shipments.
The iPhone 6S followed by its newest model, the iPhone 7 were the best-selling models this quarter.
Windows Phone experienced a QoQ decline of 35.2 percent  with a total of 974.4 thousand units shipped this quarter.
With Microsoft’s focus on business users, the decline in the consumer market is expected to continue.

The market shares of the top three mobile platforms and the remaining market during the second
quarter of each year between 2015 and 2016 are shown in (e.g. Table~\ref{tab:SmartphoneOSMarketShares}).

%--------------------TABLE--------------------------------------------------------------------
\begin{table}
\caption{Smartphone Operating System market shares from years 2015 to 2016.}
\label{tab:SmartphoneOSMarketShares}
\centering
\begin{tabular}{l l l l l}
\toprule
%\tabhead{Groups} & \tabhead{Treatment X} & \tabhead{Treatment Y} \\
\tabhead{Period} & \tabhead{Android} & \tabhead{iOS} & \tabhead{Windows Phone} & \tabhead{Others}\\
\midrule
2015Q4 & 79.6\% &18.7\% & 1.2\% &0.5\%\\
2016Q1 &83.5\%&15.4\%&0.8\%&0.4\% \\
2016Q2 &87.6\%&11.7\%&0.4\%&0.3\% \\
2016Q3 &86.8\%&12.5\%&0.3\%&0.4\%  \\
\bottomrule\\
\end{tabular}
\end{table}

%-----------------------------Android-----------------------------------------------------------
\subsection{Android}

%----- https://en.wikipedia.org/wiki/Android_(operating_system)

\paragraph{}
Initially developed by Android Inc., which Google bought in 2005, Android was unveiled in 2007, along with the founding of the Open Handset Alliance – a consortium of hardware, software, and telecommunication companies devoted to advancing open standards for mobile devices.
Beginning with the first commercial Android device in September 2008, the operating system has gone through multiple major releases, with the current version being 7.0 "Nougat", released in August 2016.
Android applications ("apps") can be downloaded from the Google Play store, which features over 2.7 million apps as of February 2017.
Android's source code is released by Google under an open source license, although most Android devices ultimately ship with a combination of free and open source and proprietary software, including proprietary software required for accessing Google services.
Android is popular with technology companies that require a ready-made, low-cost and customizable operating system for high-tech devices.

\paragraph{}
%Android is designed as a stack of various components, which  is divided into five layers. Figure 31 illustrates the different layers of the system.
Android is built on a modied Linux 2.6 series kernel that provides core system services such as security, memory management, process management, network stack and driver model. The kernel and low level tools are contained in the
bottom layer, colored red in the illustration. The basic libraries included in Android are programmed in C and C++, and are accessed through the
Android application framework.



%amit Kumar Saha. What is Android. Linux for you, (January):48{50,

\paragraph{}

The Android runtime contains a set of Java core libraries and the Dalvik
virtual machine (VM). The Dalvik VM executes les in Dalvik Executable
(.dex) format, usually transformed from Java byte code to Dalvik byte code.[16 TBD] %16 Brahler. Analysis of the android architecture. PhD thesis, Karlsruher Institut fur Technologie, 2010. URL http://www.it.iitb.ac.in/frg/wiki/images/2/20/2010_braehler-stefan_android_architecture.pdf.

\paragraph{}
Every Android application runs in its own process with its own sandboxed instance of Dalvik VM.
%Dalvik has been optimized so that a device can run multiple VMs at the same time eciently.[43] The kernel also provides an
abstraction of the underlying hardware for the rest of the software stack.[23 TBD]
%Nisarg Gandhewar and Rahila Sheikh. Google Android: An emerging software platform for mobile devices. International Journal on Computer Science and Engineering (IJCSE), (12):12{17, 2010. ISSN 0975-2297. URL http://www.enggjournals.com/ijcse/doc/003-IJCSESP24.pdf.

\paragraph{}
The application framework layer gives the developers access to the same framework Application Programming Interfaces (API) used by the core applications.
[43] The frameworks are written in Java and provide abstractions the Android libraries and the features of the Dalvik VM.[16]


\paragraph{}
Applications for Android are developed through the Android Software
Development Kit (SDK), usually with the Java programming language. The
SDK provides the API libraries and developer tools for building, testing and
debugging for Android. Development can be done in any of the current major
operating systems and an integrated development environment (IDE) of
choice, although Google recommends using Android Studio. Another common
option is to use Eclipse IDE with Android Developer Tools (ADT) plugin,
provided by Google, which integrates the Android SDK into Eclipse. The
ADT allows the developer to test the application with an Android emulator
or a connected device, and provides a graphical editor for building the user
interface (UI) of the application.[5]

\paragraph{}
The main distribution channel for Android applications is the Google Play
Store, formerly known as Android Market, where developers can publish their
applications after registration.

% image of android achiecture

%-----------------------------iOS-----------------------------------------------------------
\subsection{iOS}

% image of ios achiecture
\paragraph{}
iOSis a mobile operating system created and developed by Apple Inc. exclusively for its hardware.
It is the operating system that presently powers many of the company's mobile devices, including the iPhone, iPad, and iPod Touch. It is the second most popular mobile operating system globally after Android. iPad tablets are also the second most popular, by sales, against Android since 2013.[9]

\paragraph{}
Originally unveiled in 2007 for the iPhone, iOS has been extended to support other Apple devices such as the iPod Touch (September 2007) and the iPad (January 2010).
As of January 2017, Apple's App Store contains more than 2.2 million iOS applications, 1 million of which are native for iPads.
These mobile apps have collectively been downloaded more than 130 billion times.

%That master thesis pdf
\paragraph{}
The iOS, there are four abstraction layers: the Core OS, Core Services, Media, and Cocoa Touch layers.


\paragraph{}
%The layers provide different levels of abstraction between the applications and the underlying hardware.
The various core frameworks are written in the Objective-C programming language.
The Core OS layer contains the low-level features most of the other technologies are built upon.
Applications rarely use these technologies directly, but rather use them through the other frameworks.
However, the layer contains frameworks for features such as security, Bluetooth support and communicating with external hardware, that can be used by applications if needed.
The layer also contains the kernel environment, drivers and low-level UNIX interfaces of the operating system.
Access to the kernel and drivers is restricted to a limited set of system frameworks
and applications.[9]

\paragraph{}
The various system services used by applications are contained in the
Core Services layer. This includes technologies to support features like location,
iCloud storage, peer-to-peer services and networking. The Media layer
above it contains the graphics, audio and video technologies needed to implement
multimedia features in applications. Finally in the top layer resides
the Cocoa Touch framework, providing the key frameworks for building iOS
applications. This includes high-level programming interfaces for making
animations, networking and modifying the appearance of the application.
Cocoa Touch also handles touch-based inputs and multitasking.[8]



\paragraph{}
Building iOS applications requires using Apple's Xcode IDE on a Mac
computer running OS X 10.8 or later and iOS SDK. Xcode provides the
standard tools to code, debug and design the interface for the applications.
Generally iOS applications are written in Objective-C language.[8]

\paragraph{}
Applications for iOS are distributed to consumers exclusively through
Apple's App Store. Developers enroll in Apple Developer Program and pay
yearly fee to be able to publish applications in the App Store, and applications
go through an approval process by Apple before appearing in the
store.[8] The approval process causes longer development times, but lowers

\paragraph{}
the number of low-quality applications in the store.[17] The approval process
can pose a challenge for applications developed with various cross-platform
methods. For example, in 2010 Apple maintained that apps must be "originally
written in Objective-C, C, C++ or JavaScript" to be accepted into
the store. The restrictions have been eased since then, but applications still
sometimes get rejected for being too slow or not feeling native enough. Apple
App Store can also reject apps that download executable code or interpret
code not contained within the application archive.[32]


%-----------------------------Windows-----------------------------------------------------------
\subsection{Windows}
%Wiki windows

\paragraph{}
Windows 10 is a personal computer operating system developed and released by Microsoft as part of the Windows NT family of operating systems. It was officially unveiled in September 2014 following a brief demo at Build 2014. The first version of the operating system entered a public beta testing process in October, leading up to its consumer release on July 29, 2015.[9]

\paragraph{}
Windows 10 introduces what Microsoft described as "universal apps" expanding on Metro-style apps,
%these apps can be designed to run across multiple Microsoft product families with nearly identical code‍—‌including PCs, tablets, smartphones, embedded systems, Xbox One, Surface Hub and Mixed Reality.
%The Windows user interface was revised to handle transitions between a mouse-oriented interface and a touchscreen-optimized interface based on available input devices‍—‌particularly on 2-in-1 PCs  both interfaces include an updated Start menu which incorporates elements of Windows 7's traditional Start menu with the tiles of Windows 8.
The first release of Windows 10 also introduces a virtual desktop system, a window and desktop management feature called Task View, the Microsoft Edge web browser, support for fingerprint and face recognition login, new security features for enterprise environments, and DirectX 12 and WDDM 2.0 to improve the operating system's graphics capabilities for games.

\paragraph{}
Universal Windows Platform (UWP) apps[1] (formerly Windows Store apps and Metro-style apps)[2] are apps that can be used across all compatible Microsoft Windows devices, including personal computers (PCs), tablets, smartphones, Xbox One, Microsoft HoloLens, and Internet of Things. UWP apps are primarily purchased and downloaded via the Windows Store.[3]

\paragraph{}


%-----------------------------Tizen-----------------------------------------------------------
\subsection{Tizen}

%https://www.tizen.org/about
\paragraph{}
Tizen is an open and flexible operating system built from the ground up to address the needs of all stakeholders of the mobile and connected device ecosystem, including device manufacturers, mobile operators, application developers and independent software vendors (ISVs). Tizen is developed by a community of developers, under open source governance, and is open to all members who wish to participate.

\paragraph{}
The Tizen operating system comes in multiple profiles to serve different industry requirements. The current Tizen profiles are Tizen IVI (in-vehicle infotainment), Tizen Mobile, Tizen TV, and Tizen Wearable. In addition to that, as of Tizen 3.0, all profiles are built on top of a common, shared infrastructure called Tizen Common.

\paragraph{}
With Tizen, a device manufacturer can begin with one of these profiles and modify it to serve their own needs, or use the Tizen Common base to develop a new profile to meet the memory, processing and power requirements of any device and quickly bring it to market.

\paragraph{}
Mobile operators can work with device partners to customize the operating system and user experience to meet the needs of specific customer segments or demographics.

\paragraph{}
For application developers and ISVs, Tizen offers the power of native application development with the flexibility of unparalleled HTML5 support. Tizen also offers the potential for application developers to extend their reach to new “smart devices” running Tizen, including wearables, consumer electronics (TVs, gaming consoles, DVRs, etc.), cars and appliances.

\paragraph{}
The Tizen project resides within the Linux Foundation and is governed by a Technical Steering Group. The Technical Steering Group is the primary decision-making body for the open source project, with a focus on platform development and delivery, along with the formation of working groups to support device verticals.

\paragraph{}
The Tizen Association has been formed to guide the industry role of Tizen, including gathering of requirements, identification and facilitation of service models, and overall industry marketing and education.


\paragraph{}
Tizen provides application development tools based on the JavaScript libraries jQuery and jQuery Mobile.
Since version 2.0, a C++ native application framework is also available, based on an Open Services Platform from the Bada platform.

\paragraph{}
Samsung Releases New Preview of Visual Studio Tools for Tizen

%-----------------------------------------Mobile Applications-----------------------------------------------

\section{ Mobile Applications}


Mobile applications are consist of software/set of program that runs on a mobile device and perform certain tasks for the user.Mobile application is a new and fast developing Segment of the global Information and Communication Technology.
\paragraph{}
Mobile Screens are small, But Mobile apps are big, and life as we know it is on its head again.
In a world that's increasingly social and open, mobile apps play a vital role, and Today we have changed the focus from what's on the Web, to the apps on our mobile device. Mobile apps are very imperative. But where do we  start?  There are many factors that play a part in your mobile strategy, such as your team’s development skills, required device functionality, the importance of security, offline capability, interoperability, etc., that must be taken into account. Finally it’s not just a question of what mobile application  will do, but how we will reach there. get it there.

\paragraph{}
Mobile applications come in two distinct formats: native apps and web apps. Due to differences in their underlying technology, each approach has inherent advantages and drawbacks.


\subsection{Native Applications}

A native mobile app is built specifically for a particular device and its operating system. Unlike a web app that is accessed over the internet, a native app is downloaded from a web app store and installed on the device. Native apps are written in Java, Objective C, or some other programming language. This is changing with HTML5, but functionality is inconsistent and incomplete.



\paragraph{}
% Ref Mobile Web Apps vs. Mobile Native Apps: How to Make the Right Choice

Native apps have a major advantage over web applications the ability to leverage device-specific hardware and software. This means that native apps can take advantage of the latest technology available on mobile devices and can integrate with on-board apps such as the calendar, contacts, and email. However, this is a double-edged sword: while mobile technology is wildly popular, it is also constantly changing and highly fragmented. This makes the task of keeping up with the pace of emerging technology onerous and costly, especially on multiple platforms.



%--------------------TABLE--------------------------------------------------------------------
\begin{table}
\caption{Native App Development across different platform  }
\label{tab:Features Supported Across Platform}
\centering
\begin{tabular}{@{}lllll@{}}
\toprule

\tabhead{} & \tabhead{Android} & \tabhead{iOS} & \tabhead{Windows 10} & \tabhead{Tizen}\\
\midrule


\hline
\multicolumn{1}{l}{{\bf \shortstack [l] { Programming \\ languages } }}  & 	JAVA	& 	Objective-C,Swift & XAML,C\#	& 	C,C++ \\
\hline
\multicolumn{1}{l}{{\bf \shortstack [l] { Integrated\\ Development \\ Environment }  }}	& 	Android Studio	& 	Xcode	& 	Visual Studio & Tizen Studio	\\
\hline
\multicolumn{1}{l}{{\bf \shortstack [l]{Software \\  Development Kit }  }} & 	Android SDK	& 	iOS SDK	&  Windows SDK	& Tizen SDK	\\
\hline
\multicolumn{1}{l}{{\bf \shortstack [l] {App Distribution} }}	& 	Google play	& 	Appstore	& 	Windows Store & Tizen Store	\\
\hline
\multicolumn{1}{l}{{\bf Community }} & 	Very Good 	& 	Very Good	& 	Good  &  Limited \\
\hline


\bottomrule\\
\end{tabular}
\end{table}


%----------------------------------------------------------------------------------------


\subsection{Mobile Web Apps}

A mobile web app is a web application formatted for smartphones and tablets, and accessed through the mobile device’s web browser. Like a traditional web application, a mobile web app is built with three core technologies: HTML (defines static text and images), CSS (defines style and presentation), and JavaScript (defines interactions and animations).



\paragraph{}

Since web apps are browser-based, they’re intended to be platform and device independent, able to run on any web-enabled smartphone or tablet. A mobile web app is normally downloaded from a central web server each time it is run, although apps built using HTML5 (described below) can also run on the mobile device for offline use.

\paragraph{}
However, significant limitations, especially for enterprise mobile, are offline storage and security. While you can implement a semblance of offline capability by caching files on the device, it just isn't a very good solution. Although the underlying database might be encrypted, it’s not as well segmented as a native keychain encryption that protects each app with a developer certificate. Also, if a web app with authentication is launched from the desktop, it will require users to enter their credentials every time the app it is sent to the background. This is a lousy experience for the user. In general, implementing even trivial security measures on a native platform can be complex tasks for a mobile Web developer. Therefore, if security is of the utmost importance, it can be the deciding factor on which mobile technology you choose.

% "Native, HTML5, or Hybrid: Understanding Your Mobile Application Development Options"


\subsection{Hybrid Mobile Applications}

Hybrid development combines the best (or worst) of both the native and HTML5 worlds. We define hybrid as a web app, primarily built using HTML5 and JavaScript, that is then wrapped inside a thin native container that provides access to native platform features. PhoneGap is an example of the most popular container for creating hybrid mobile apps.

\paragraph{}
For the most part, hybrid apps provide the best of both worlds. Existing web developers that have become gurus at optimizing JavaScript, pushing CSS to create beautiful layouts, and writing compliant HTML code that works on any platform can now create sophisticated mobile applications that don’t sacrifice the cool native capabilities. In certain circumstances, native developers can write plugins for tasks like image processing, but in cases like this, the devil is in the details.

\paragraph{}

On iOS, the embedded web browser or the UIWebView is not identical to the Safari browser. While the differences are minor, they can cause debugging headaches. That’s why it pays off to invest in popular frameworks that have addressed all of the limitations.

\paragraph{}
You know that native apps are installed on the device, while HTML5 apps reside on a Web server, so you might be wondering if hybrid apps store their files on the device or on a server? Yes. In fact there are two ways to implement a hybrid app.


\paragraph{}

\begin{itemize}
\item Local - You can package HTML and JavaScript code inside the mobile application binary, in a manner similar to the structure of a native application. In this scenario you use REST APIs to move data back and forth between the device and the cloud.
\item Server - Alternatively you can implement the full web application from the server (with optional caching for better performance), simply using the container as a thin shell over the UIWebview.
\end{itemize}

%Netflix has a really cool app that uses the same code base for running the UI on all devices: tablets, %phones, smart TVs, DVD players, refrigerators, and cars. While most people have no idea, nor care, %how the app is implemented, you’ll be interested to know they can change the interface on the fly %or conduct A/B testing to determine the optimal user interactions. The guts of decoding and %streaming videos are delegated to the native layer for best performance, so it’s a fast, seemingly % native app, that really does provide the best of both worlds.

\subsection{Conclusion}

Mobile development is a constantly moving target. Every six months, there’s a new mobile operating system, with unique features only accessible with native APIs. The containers bring those to hybrid apps soon thereafter, with the web making tremendous leaps every few years. Based on current technology, one of the scenarios examined in this article is bound to suit your needs. Let's sum those up in the following table:


%--------------------TABLE--------------------------------------------------------------------
\begin{table}
\caption{The Mobile App Comparison Chart:  Native vs. Mobile Web vs.Hybrid }
\label{tab:Features Supported Across Platform}
\centering
\begin{tabular}{l l l l}
\toprule

\tabhead{} & \tabhead{Native} & \tabhead{HTML5} & \tabhead{Hybrid}\\
\midrule

\multicolumn{1}{l}{{\bf App Features}} \\

\hline
Graphics	& 	Native APIs	& 	HTML, Canvas, SVG	& 	HTML, Canvas, SVG	\\
Performance	& 	Fast	& 	Slow	& 	Slow	\\
Native look and feel	& 	Native	& 	Emulated	& 	Emulated	\\
Distribution	& 	Appstore	& 	Web	& 	Appstore	\\
\hline

\multicolumn{1}{l}{{\bf Device Access}} \\

\hline
Camera	& 	Yes	& 	No	& 	Yes	\\
Notifications	& 	Yes	& 	No	& 	Yes	\\
Contacts, calendar	& 	Yes	& 	No	& 	Yes	\\
Offline storage	& 	Secure file storage	& Shared SQL	& 	Secure file, shared SQL	\\
Geolocation	& 	Yes	& 	Yes	& 	Yes	\\
\hline

\multicolumn{1}{l}{{\bf Gestures}} \\
\hline
Swipe	& 	Yes	& 	Yes	& 	Yes	\\
Pinch, spread	& 	Yes	& 	No	& 	Yes	\\
\hline

\multicolumn{1}{l}{{\bf Connectivity}} \\
\hline
Connectivity	& 	Online and offline	& 	online	& 	Online and offline	\\
\hline

\multicolumn{1}{l}{{\bf Developer Skills}} \\
\hline
language & 	Objective C, Java	& 	HTML5, CSS, Javascript	& 	HTML5, CSS, Javascript\\
\hline

\bottomrule\\
\end{tabular}
\end{table}


%----------------------------------------------------------------------------------------
%----------------------------------------------------------------------------------------
\section{Cross-Platform Development Tools}


\subsection{PhoneGap}

PhoneGap is the open source framework that gets you building amazing mobile apps using web technology.

\paragraph{}
Developing your mobile app using HTML, CSS and Javascript doesn’t mean that you have to give up on native functionality that makes mobile devices so extraordinary.
PhoneGap gives you access to all of the native device APIs (like camera, GPS, accelerometer and more),
so that the app you build with web tech behaves just like a native app. but still have some limitations.

\paragraph{}
Since the front end of the application is built in JavaScript, it causes a number of limitations.

%referance
%http://stackoverflow.com/questions/8212547/what-are-the-limitations-and-disadvantages-of-phonegap-or-html5-as-compared-to-t
\begin{itemize}


  \item \textbf{Data processing:}  Native languages are much faster than JavaScript for data processing on the device.

  \item \textbf{Background processing:}  A large number of applications rely on background threads to provide a smooth user experience: calculating the GPS positions in the background, for example. PhoneGap APIs are built using JavaScript which is not multi-threaded and hence do not support background processing.

  \item \textbf{Access advanced native functionality:}   A number of native APIs are not yet supported by PhoneGap’s APIs.

  \item \textbf{Complex Business Logic: }  A number of applications such as enterprise applications are quite complex. In this scenario it is simply better to have a certain amount of native code.

  \item \textbf{Advanced Graphics:}  Advanced Graphics: Apps that use advanced graphics which can only be accessed using third-party libraries are best done natively.

\end{itemize}


%----------------------------------------------------------------------------------------
%Referance: https://wiki.appcelerator.org/pages/viewpage.action?pageId=29004882
\subsection{Titanium}

The Titanium SDK helps you to build native cross-platform mobile application using JavaScript and the Titanium API,
which abstracts the native APIs of the mobile platforms.  Titanium empowers you to create immersive, full-featured applications, featuring over 80\% code reuse across mobile apps.
Appcelerator licenses Titanium under the Apache 2 license and is free for both personal and commercial use.

\paragraph{}
Titanium's unique trait among the various available cross-platform mobile solutions is that it creates truly native apps.
This is in contrast to web-based solutions that deliver functionality via an enhanced web view.
Titanium, not wanting to be limited by the native web view, has engaged in a much deeper integration with the underlying platforms.
This gives Titanium developers the ability to leverage native UI components and  services, as well as near-native performance.

\paragraph{}
Since the front end of the application is built in JavaScript, it causes a number of limitations.

%referance
% https://enricoangelini.com/2012/5-pros-and-cons-of-appcelerators-titanium/
\begin{itemize}


 \item \textbf{Increasing complexity:}  The development complexities (and costs) rise more than proportionally to application complexity increases. The more complex your applications become, the more often you’ll have to deal with, on the one hand technical issues (random crashes, weird behaviors, annoying bugs, etc.), on the other hand a greater effort (code organization, MVC separation, multi-device support, multi-platform support, code readability, etc.)

 \item \textbf{No Freemium:}  Appcelerator provides StoreKit, a module to enable In-App Purchase to Apple’s App Store, but it’s a pain.
 Buggy, poorly documented and it seems to work only partially. Too unstable for production use. Having to renounce the freemium pricing model (apps that are free to download,
 but require an in-app purchase to be expanded) is not just a minor inconvenience since 72\% of total App Store revenue comes from apps featuring in-app purchases.

 \item \textbf{Toolkit pain:}   At first there was only Titanium Developer but since last June there is Titanium Studio, an Eclipse-based IDE built on a modified version of Aptana that allows you to manage projects, test your mobile apps in the simulator or on device and automate app packaging. First of all,
 I sincerely hate Eclipse, yeah, Eclipse is free and the best open source IDE there is, but offers a very poor IDE experience.
 Most importantly Titanium Studio can go “crazy”, encounter weird glitches, stop printing console messages,
  but the worst thing is when the build process start to ignore changes. You have to continually clean your project every time you make changes or restart with a brand new project. A productivity tool that is uncomfortable and unstable is not a productivity tool and a development tool that is unproductive is not a development tool.

 \item \textbf{Flexibility limitations: }  All that glitters is not gold. At beginning you’ll love the well-defined Titanium API and you will probably love it even more every time you discover a simple property to enable behaviors that would require several lines of code on Xcode. But sooner or later you will face strange bugs and limitations. For example, if you want to apply a cell background gradient to a grouped table (a very common and easy task with Objective-C) you get that the grouped table becomes plain and the gradient color makes the table slow when scrolling, and you will have to use images… So at first you will save a lot of time but as more complex the project grows you’ll lose the saved time in fixing and workarounds.

 \item \textbf{Laggy:}   Obviously you can have the most smooth, fast and comfortable user experience possible only with apps developed with a native development environment. This is an obvious observation, but which cannot be omitted. Keep in mind that a Titanium application is the result of an automatic conversion process from web code to native code. Animations are noticeably laggy and apps are not responsive when return from the background. This is particularly evident with Android devices, less evident with iOS devices (especially those with A5 processor).
\end{itemize}



\subsection{Xamarin}
%changes to be made here soon
%refer:https://www.quora.com/What-are-the-drawbacks-of-developing-your-mobile-apps-on-Xamarin      https://en.wikipedia.org/wiki/Xamarin
%http://www.sphinx-solution.com/blog/comparison-of-cross-platform-app-development-frameworks-ionic-vs-phonegap-vs-xamarin-vs-titanium/
Launched in 2011, Xamarin is a mono framework used for cross-platform app development. Xamarin brings open source .NET to mobile development, enabling every developer to build truly native apps for any device in C\# and F\#.
We’re excited for your contributions in continuing our mission to make it fast, easy, and fun to build great mobile apps.

\paragraph{}

Xamarin is best for building applications using C\# programming language running on .NET Common Language Infrastructure (CLI) (often called Microsoft .NET).
\paragraph{}
Over 1.5 million developers were using Xamarin's products in more than 120 countries around the world as of May 2015.
It is widely used for communicating with the Application Program Interface (API) of common mobile device functions like contacts, camera, and geolocation for android, iOS, and Windows operating systems.
It allows developers to use almost 100 \% native libraries of both Android and iOS, With a C\#-shared codebase, developers can use Xamarin tools to write native Android, iOS, and Windows apps with native user interfaces and share code across multiple platforms, including Windows and macOS.

\paragraph{}
On February 24, 2016, Microsoft announced it had signed a definitive agreement to acquire Xamarin.
