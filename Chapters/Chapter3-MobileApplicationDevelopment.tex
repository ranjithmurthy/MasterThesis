% Chapter 3 % Not completed on going still
% !TEX root = ../main.tex
\chapter{Mobile Application Development} % Main chapter title

\label{Chapter3} % For referencing the chapter elsewhere, use \ref{Chapter1}

%----------------------------------------------------------------------------------------

% Define some commands to keep the formatting separated from the content
% \newcommand{\keyword}[1]{\textbf{#1}}
% \newcommand{\tabhead}[1]{\textbf{#1}}
% \newcommand{\code}[1]{\texttt{#1}}
% \newcommand{\file}[1]{\texttt{\bfseries#1}}
% \newcommand{\option}[1]{\texttt{\itshape#1}}


%----------------------------------------------------------------------------------------


%-----------------------------------------Mobile Applications-----------------------------------------------

\section{ Mobile Applications}


Mobile applications are consist of software/set of program that runs on a mobile device and perform certain tasks for the user.Mobile application is a new and fast developing Segment of the global Information and Communication Technology.
\paragraph{}
Mobile Screens are small, But Mobile apps are big, and life as we know it is on its head again.
In a world that's increasingly social and open, mobile apps play a vital role, and Today we have changed the focus from what's on the Web, to the apps on our mobile device. Mobile apps are very imperative. But where do we  start?  There are many factors that play a part in your mobile strategy, such as your team’s development skills, required device functionality, the importance of security, offline capability, interoperability, etc., that must be taken into account. Finally it’s not just a question of what mobile application  will do, but how we will reach there. get it there.

\paragraph{}
Mobile applications come in two distinct formats: native apps and web apps. Due to differences in their underlying technology, each approach has inherent advantages and drawbacks.


\subsection{Native Applications}

A native mobile app is built specifically for a particular device and its operating system. Unlike a web app that is accessed over the internet, a native app is downloaded from a web app store and installed on the device. Native apps are written in Java, Objective C, or some other programming language. This is changing with HTML5, but functionality is inconsistent and incomplete.



\paragraph{}
% Ref Mobile Web Apps vs. Mobile Native Apps: How to Make the Right Choice

Native apps have a major advantage over web applications the ability to leverage device-specific hardware and software. This means that native apps can take advantage of the latest technology available on mobile devices and can integrate with on-board apps such as the calendar, contacts, and email. However, this is a double-edged sword: while mobile technology is wildly popular, it is also constantly changing and highly fragmented. This makes the task of keeping up with the pace of emerging technology onerous and costly, especially on multiple platforms.






%----------------------------------------------------------------------------------------


\subsection{Mobile Web Apps}

A mobile web app is a web application formatted for smartphones and tablets, and accessed through the mobile device’s web browser. Like a traditional web application, a mobile web app is built with three core technologies: HTML (defines static text and images), CSS (defines style and presentation), and JavaScript (defines interactions and animations).



\paragraph{}

Since web apps are browser-based, they’re intended to be platform and device independent, able to run on any web-enabled smartphone or tablet. A mobile web app is normally downloaded from a central web server each time it is run, although apps built using HTML5 (described below) can also run on the mobile device for offline use.

\paragraph{}
However, significant limitations, especially for enterprise mobile, are offline storage and security. While you can implement a semblance of offline capability by caching files on the device, it just isn't a very good solution. Although the underlying database might be encrypted, it’s not as well segmented as a native keychain encryption that protects each app with a developer certificate. Also, if a web app with authentication is launched from the desktop, it will require users to enter their credentials every time the app it is sent to the background. This is a lousy experience for the user. In general, implementing even trivial security measures on a native platform can be complex tasks for a mobile Web developer. Therefore, if security is of the utmost importance, it can be the deciding factor on which mobile technology you choose.

% "Native, HTML5, or Hybrid: Understanding Your Mobile Application Development Options"


\subsection{Hybrid Mobile Applications}

Hybrid development combines the best (or worst) of both the native and HTML5 worlds. We define hybrid as a web app, primarily built using HTML5 and JavaScript, that is then wrapped inside a thin native container that provides access to native platform features. PhoneGap is an example of the most popular container for creating hybrid mobile apps.

\paragraph{}
For the most part, hybrid apps provide the best of both worlds. Existing web developers that have become gurus at optimizing JavaScript, pushing CSS to create beautiful layouts, and writing compliant HTML code that works on any platform can now create sophisticated mobile applications that don’t sacrifice the cool native capabilities. In certain circumstances, native developers can write plugins for tasks like image processing, but in cases like this, the devil is in the details.

\paragraph{}

On iOS, the embedded web browser or the UIWebView is not identical to the Safari browser. While the differences are minor, they can cause debugging headaches. That’s why it pays off to invest in popular frameworks that have addressed all of the limitations.

\paragraph{}
You know that native apps are installed on the device, while HTML5 apps reside on a Web server, so you might be wondering if hybrid apps store their files on the device or on a server? Yes. In fact there are two ways to implement a hybrid app.


\paragraph{}

\begin{itemize}
\item Local - You can package HTML and JavaScript code inside the mobile application binary, in a manner similar to the structure of a native application. In this scenario you use REST APIs to move data back and forth between the device and the cloud.
\item Server - Alternatively you can implement the full web application from the server (with optional caching for better performance), simply using the container as a thin shell over the UIWebview.
\end{itemize}

%Netflix has a really cool app that uses the same code base for running the UI on all devices: tablets, %phones, smart TVs, DVD players, refrigerators, and cars. While most people have no idea, nor care, %how the app is implemented, you’ll be interested to know they can change the interface on the fly %or conduct A/B testing to determine the optimal user interactions. The guts of decoding and %streaming videos are delegated to the native layer for best performance, so it’s a fast, seemingly % native app, that really does provide the best of both worlds.

\subsection{Conclusion}

Mobile development is a constantly moving target. Every six months, there’s a new mobile operating system, with unique features only accessible with native APIs. The containers bring those to hybrid apps soon thereafter, with the web making tremendous leaps every few years. Based on current technology, one of the scenarios examined in this article is bound to suit your needs. Let's sum those up in the following table:


%--------------------TABLE--------------------------------------------------------------------
\begin{table}
\caption{The Mobile App Comparison Chart:  Native vs. Mobile Web vs.Hybrid }
\label{tab:Features Supported Across Platform}
\centering
\begin{tabular}{l l l l}
\toprule

\tabhead{} & \tabhead{Native} & \tabhead{HTML5} & \tabhead{Hybrid}\\
\midrule

\multicolumn{1}{l}{{\bf App Features}} \\

\hline
Graphics	& 	Native APIs	& 	HTML, Canvas, SVG	& 	HTML, Canvas, SVG	\\
Performance	& 	Fast	& 	Slow	& 	Slow	\\
Native look and feel	& 	Native	& 	Emulated	& 	Emulated	\\
Distribution	& 	Appstore	& 	Web	& 	Appstore	\\
\hline

\multicolumn{1}{l}{{\bf Device Access}} \\

\hline
Camera	& 	Yes	& 	No	& 	Yes	\\
Notifications	& 	Yes	& 	No	& 	Yes	\\
Contacts, calendar	& 	Yes	& 	No	& 	Yes	\\
Offline storage	& 	Secure file storage	& Shared SQL	& 	Secure file, shared SQL	\\
Geolocation	& 	Yes	& 	Yes	& 	Yes	\\
\hline

\multicolumn{1}{l}{{\bf Gestures}} \\
\hline
Swipe	& 	Yes	& 	Yes	& 	Yes	\\
Pinch, spread	& 	Yes	& 	No	& 	Yes	\\
\hline

\multicolumn{1}{l}{{\bf Connectivity}} \\
\hline
Connectivity	& 	Online and offline	& 	online	& 	Online and offline	\\
\hline

\multicolumn{1}{l}{{\bf Developer Skills}} \\
\hline
language & 	Objective C, Java	& 	HTML5, CSS, Javascript	& 	HTML5, CSS, Javascript\\
\hline

\bottomrule\\
\end{tabular}
\end{table}


%----------------------------------------------------------------------------------------
%----------------------------------------------------------------------------------------
\section{Cross-Platform Development Tools}


\subsection{PhoneGap}

PhoneGap is the open source framework that gets you building amazing mobile apps using web technology.

\paragraph{}
Developing your mobile app using HTML, CSS and Javascript doesn’t mean that you have to give up on native functionality that makes mobile devices so extraordinary.
PhoneGap gives you access to all of the native device APIs (like camera, GPS, accelerometer and more),
so that the app you build with web tech behaves just like a native app. but still have some limitations.

\paragraph{}
Since the front end of the application is built in JavaScript, it causes a number of limitations.

%referance
%http://stackoverflow.com/questions/8212547/what-are-the-limitations-and-disadvantages-of-phonegap-or-html5-as-compared-to-t
\begin{itemize}


  \item \textbf{Data processing:}  Native languages are much faster than JavaScript for data processing on the device.

  \item \textbf{Background processing:}  A large number of applications rely on background threads to provide a smooth user experience: calculating the GPS positions in the background, for example. PhoneGap APIs are built using JavaScript which is not multi-threaded and hence do not support background processing.

  \item \textbf{Access advanced native functionality:}   A number of native APIs are not yet supported by PhoneGap’s APIs.

  \item \textbf{Complex Business Logic: }  A number of applications such as enterprise applications are quite complex. In this scenario it is simply better to have a certain amount of native code.

  \item \textbf{Advanced Graphics:}  Advanced Graphics: Apps that use advanced graphics which can only be accessed using third-party libraries are best done natively.

\end{itemize}


%----------------------------------------------------------------------------------------
%Referance: https://wiki.appcelerator.org/pages/viewpage.action?pageId=29004882
\subsection{Titanium}

The Titanium SDK helps you to build native cross-platform mobile application using JavaScript and the Titanium API,
which abstracts the native APIs of the mobile platforms.  Titanium empowers you to create immersive, full-featured applications, featuring over 80\% code reuse across mobile apps.
Appcelerator licenses Titanium under the Apache 2 license and is free for both personal and commercial use.

\paragraph{}
Titanium's unique trait among the various available cross-platform mobile solutions is that it creates truly native apps.
This is in contrast to web-based solutions that deliver functionality via an enhanced web view.
Titanium, not wanting to be limited by the native web view, has engaged in a much deeper integration with the underlying platforms.
This gives Titanium developers the ability to leverage native UI components and  services, as well as near-native performance.

\paragraph{}
Since the front end of the application is built in JavaScript, it causes a number of limitations.

%referance
% https://enricoangelini.com/2012/5-pros-and-cons-of-appcelerators-titanium/
\begin{itemize}


 \item \textbf{Increasing complexity:}  The development complexities (and costs) rise more than proportionally to application complexity increases. The more complex your applications become, the more often you’ll have to deal with, on the one hand technical issues (random crashes, weird behaviors, annoying bugs, etc.), on the other hand a greater effort (code organization, MVC separation, multi-device support, multi-platform support, code readability, etc.)

 \item \textbf{No Freemium:}  Appcelerator provides StoreKit, a module to enable In-App Purchase to Apple’s App Store, but it’s a pain.
 Buggy, poorly documented and it seems to work only partially. Too unstable for production use. Having to renounce the freemium pricing model (apps that are free to download,
 but require an in-app purchase to be expanded) is not just a minor inconvenience since 72\% of total App Store revenue comes from apps featuring in-app purchases.

 \item \textbf{Toolkit pain:}   At first there was only Titanium Developer but since last June there is Titanium Studio, an Eclipse-based IDE built on a modified version of Aptana that allows you to manage projects, test your mobile apps in the simulator or on device and automate app packaging. First of all,
 I sincerely hate Eclipse, yeah, Eclipse is free and the best open source IDE there is, but offers a very poor IDE experience.
 Most importantly Titanium Studio can go “crazy”, encounter weird glitches, stop printing console messages,
  but the worst thing is when the build process start to ignore changes. You have to continually clean your project every time you make changes or restart with a brand new project. A productivity tool that is uncomfortable and unstable is not a productivity tool and a development tool that is unproductive is not a development tool.

 \item \textbf{Flexibility limitations: }  All that glitters is not gold. At beginning you’ll love the well-defined Titanium API and you will probably love it even more every time you discover a simple property to enable behaviors that would require several lines of code on Xcode. But sooner or later you will face strange bugs and limitations. For example, if you want to apply a cell background gradient to a grouped table (a very common and easy task with Objective-C) you get that the grouped table becomes plain and the gradient color makes the table slow when scrolling, and you will have to use images… So at first you will save a lot of time but as more complex the project grows you’ll lose the saved time in fixing and workarounds.

 \item \textbf{Laggy:}   Obviously you can have the most smooth, fast and comfortable user experience possible only with apps developed with a native development environment. This is an obvious observation, but which cannot be omitted. Keep in mind that a Titanium application is the result of an automatic conversion process from web code to native code. Animations are noticeably laggy and apps are not responsive when return from the background. This is particularly evident with Android devices, less evident with iOS devices (especially those with A5 processor).
\end{itemize}



\subsection{Xamarin}
%changes to be made here soon
%refer:https://www.quora.com/What-are-the-drawbacks-of-developing-your-mobile-apps-on-Xamarin      https://en.wikipedia.org/wiki/Xamarin
%http://www.sphinx-solution.com/blog/comparison-of-cross-platform-app-development-frameworks-ionic-vs-phonegap-vs-xamarin-vs-titanium/
Launched in 2011, Xamarin is a mono framework used for cross-platform app development. Xamarin brings open source .NET to mobile development, enabling every developer to build truly native apps for any device in C\# and F\#.
We’re excited for your contributions in continuing our mission to make it fast, easy, and fun to build great mobile apps.

\paragraph{}

Xamarin is best for building applications using C\# programming language running on .NET Common Language Infrastructure (CLI) (often called Microsoft .NET).
\paragraph{}
Over 1.5 million developers were using Xamarin's products in more than 120 countries around the world as of May 2015.
It is widely used for communicating with the Application Program Interface (API) of common mobile device functions like contacts, camera, and geolocation for android, iOS, and Windows operating systems.
It allows developers to use almost 100 \% native libraries of both Android and iOS, With a C\#-shared codebase, developers can use Xamarin tools to write native Android, iOS, and Windows apps with native user interfaces and share code across multiple platforms, including Windows and macOS.

\paragraph{}
On February 24, 2016, Microsoft announced it had signed a definitive agreement to acquire Xamarin.

\paragraph{}
Pros of Using Xamarin for Development

%referance
% https://www.altexsoft.com/blog/mobile/the-good-and-the-bad-of-xamarin-mobile-development/

\begin{itemize}


 \item \textbf{One Technology Stack to Code for All Platforms:}
 Xamarin uses C\# complemented with .Net framework to create apps for any mobile platform. Thus, you can reuse up to 96 percent of the source code speeding up the engineering cycle. Xamarin also does not require switching between the development environments as it works with both Xamarin IDE (for Mac) or Visual Studio (for Windows). Although many developers argued about the quality of support provided by both IDEs, Xamarin Visual Studio integration has been largely improved since the company’s acquisition by Microsoft. The cross-platform development tools are provided as a built-in part of the IDE at no additional cost.


 \item \textbf{ Performance Close to Native:}  AUnlike traditional hybrid solutions, based on the web technologies, a cross-platform app built with Xamarin, can still be classified as native. The performance metrics are comparable to those of Java for Android (as explained here) and Objective-C or Swift for native iOS app development. Moreover, the efficiency is constantly being improved to fully match the standards of native development. Xamarin platform offers a complete solution for testing and tracking the app’s performance. Its’ Xamarin Test Cloud paired with Xamarin Test Recorder tool allow you to run automated UI tests and identify performance issues before the release. However, this service is provided at an additional fee.

 \item \textbf{Native User Experiences:}

 Xamarin allows you to create flawless experiences using platform-specific UI elements. Simple cross-platform apps for iOS, Android or Windows are built using Xamarin.Forms tool, which converts app UI components into the platform-specific interface elements at runtime. As the use of Xamarin.Forms significantly increases the speed of app development, it is a great option for business-oriented projects. Yet, there might be a slight decline in performance due to the extra abstraction layer. For custom app UI and higher performance you can still use Xamarin.iOS and Xamarin.Android separately to ensure excellent results.


 \item \textbf{Full Hardware Support: }

 With Xamarin, your solution gets native-level app functionality. It eliminates all hardware compatibility issues, using plugins and specific APIs, to work with common devices functionality across the platforms. Along with the access to platform-specific APIs, Xamarin supports linking with native libraries. This allows for better customization and native-level functionality with little overhead.


\end{itemize}

\paragraph{}
Cons of Using Xamarin for Development

\begin{itemize}


 \item \textbf{Expensive Xamarin License:}

 Business subscription comes at the annual fee of \$999 per developer, per device platform, which might seem a little too high if you plan to create only one small app.
 For example, it will cost you almost \$10,000 annually to run a team of five engineers, each building apps for iOS and Android.
 However, if you are going to build other cross-platform mobile solutions in the future or provide Xamarin app development services,
 Xamarin license would be a good investment compared to the development cost of native apps.



 \item \textbf{ Slightly Delayed Support for the Latest Platform Updates:}

 This depends completely on the Xamarin developer team. It’s impossible for third-party tools to provide the immediate support for the latest iOS and Android releases: it takes some time to implement the changes and/or introduce new plugins, etc.
 Although Xamarin claims to provide same-day support, there still might be some delays.


 \item \textbf{Limited Access to Open Source Libraries:}

 Native development makes extensive use of open source technologies.
 With Xamarin, you have to use only the components provided by the platform and some .Net open source resources, facing both developers and consumers. While the choice is not quite as rich as it is for Android and iOS mobile app development, the Xamarin Components provide thousands of custom UI controls, various charts, graphs, themes, and other powerful features that can be added to an app in just a few clicks.
 This includes built-in payment processing (such as Stripe), Beacons and wearables integration, out of the box push notification services, cloud storage solutions, multimedia streaming capabilities and much more.

 \item \textbf{ Xamarin Ecosystem Problems: }

 Obviously, Xamarin community is significantly smaller than those of iOS or Android. T
 hus, finding an experienced Xamarin developer could be a challenge. Although the platform is growing its following fueled by the support from Microsoft. Based on the info from different sources, Xamarin community makes 10 percent of the global mobile development society.
 Despite the fact that the number of Xamarin engineers does not compare to iOS or Android native communities, the platform provides extensive support to its developers.
 Namely, there is a dedicated educational platform, Xamarin University, that provides resources and practical training for those who are new to this technology. Using this support, the learning curve for an experienced C\#.Net engineer is minimal.

\end{itemize}





\begin{table}
\caption{The Mobile SDK Xamarin vs:  Native vs Hybrid }
\label{tab:Mobile SDK Features Supported Across Platform}
\centering
\begin{tabular}{l l l l}

  \toprule

  \tabhead{}
  & \tabhead{Xamarin}
  & \tabhead{Native}
  & \tabhead{Hybrid}\\
  \midrule


\bf \shortstack [l] { Technology  \\ Stack }
& \shortstack [l] { C\# ,\\.net framework,\\Native libraries}
& \shortstack [l] { Different \\ Technology \\ stacks for \\ each platform }
& \shortstack [l] { JavaScript, \\ Html5, \\CSS }\\



\hline
\bf \shortstack [l] { Development\\ Cost }
& \shortstack [l] { Low to reasonable }
& \shortstack [l] {Expensive}
&  \shortstack [l] {reasonable} \\
\hline




\hline

\bf \shortstack [l] { Code  \\ Sharing }
& \shortstack [l] { Yes  \\ ( Upto 96\%)}
& \shortstack [l] {No}
&  \shortstack [l] {Yes (100\%)} \\
\hline



\bf \shortstack [l] { UI/UX }
& \shortstack [l] {Completely \\Native Ui }
& \shortstack [l] {Completely \\Native Ui }
& \shortstack [l] { Limited \\ Native Ui capabilties }\\
\hline

  \bf \shortstack [l] { Performance }
	& \shortstack [l] { Good,\\ Close to Native}
  & \shortstack [l] {Excellent}
  & \shortstack [l] {Medium}\\
\hline

\hline
\bf \shortstack [l] { Hardware \\ Capabilites }
& \shortstack [l] { Highly supported \\ Xamarin uses  \\ platform specific APIs}
& \shortstack [l] { High Native \\  Tools have \\  completely  support }
& \shortstack [l]{ Medium \\  with third party \\  API and plugins} . \\

\hline

\bf \shortstack [l] {Time to market}
& \shortstack [c] { Fast}
& \shortstack [c] { Time consuming}
& \shortstack [c] { Faster}\\


\bottomrule\\
\end{tabular}
\end{table}



\subsection{Conclusion}
% https://www.altexsoft.com/blog/mobile/the-good-and-the-bad-of-xamarin-mobile-development/
%The Good and The Bad of Xamarin Mobile Development
When comparing the pros and cons, the listed drawbacks are usually considered to be a collateral damage.
Most business owners choose Xamarin mobile app development platform as it decreases the time to market and engineering cost, by sharing the code and using a single technology stack.
Yet the purpose of the app and its target audience might be an even more important factor to consider.

\paragraph{}
Based on our team’s experience, the best use-case for Xamarin is enterprise mobile solutions.
With standard UI which covers 90 percent of the projects, all the core product logic can be easily shared across the platforms.
 Hence, platform customization will only take 5-10 percent of the engineering effort.

\paragraph{}
In case of consumer-facing apps with heavy UI, the amount of shared code decreases drastically.
Thus, Xamarin cross-platform development loses its major benefit and might equal in time and cost to native solutions.

\paragraph{}
However, if you are looking for a Xamarin alternative to build a cross-platform mobile app, you might be disappointed.
While the most widely used cross-platform mobile development tools are PhoneGap/Apache Cordova, Ionic Framework, Appcelerator/Titanium,
they rely primarily on web technologies, such as HTM5 or JavaScript.
That is why none of these tools can have the same level of performance and native functionality that Xamarin offers


%----------------------------------------------------------------------------------------
%	Mobile Development SDLC
%----------------------------------------------------------------------------------------
\section{Mobile Development SDLC}

\section{Overview}


Building mobile applications can be as easy as opening up the IDE, throwing something together, doing a quick bit of testing, and submitting to an App Store – all done in an afternoon. Or it can be an extremely involved process that involves rigorous up-front design, usability testing, QA testing on thousands of devices, a full beta lifecycle, and then deployment a number of different ways.

In this document, we’re going to take a thorough introductory examination of building mobile applications, including:

Process – The process of software development is called the Software Development Lifecycle (SDLC). We’ll examine all phases of the SDLC with respect to mobile application development, including: Inspiration, Design, Development, Stabilization, Deployment, and Maintenance.
Considerations – There are a number of considerations when building mobile applications, especially in contrast to traditional web or desktop applications. We’ll examine these considerations and how they affect mobile development.
This document is intended to answer fundamental questions about mobile app development, for new and experienced application developers alike. It takes a fairly comprehensive approach to introducing most of the concepts you’ll run into during the entire Software Development Lifecycle (SDLC).


The lifecycle of mobile development is largely no different than the SDLC for web or desktop applications. As with those, there are usually 5 major portions of the process:

\begin{itemize}


 \item \textbf{Inception :}  All apps start with an idea. That idea is usually refined into a solid basis for an application.
 \item \textbf{Design :} Design  The design phase consists of defining the app’s User Experience (UX) such as what the general layout is, how it works, etc., as well as turning that UX into a proper User Interface (UI) design, usually with the help of a graphic designer.
 \item \textbf{Development:} Usually the most resource intensive phase, this is the actual building of the application.
 \item \textbf{Stabilization :} When development is far enough along, QA usually begins to test the application and bugs are fixed. Often times an application will go into a limited beta phase in which a wider user audience is given a chance to use it and provide feedback and inform changes.
 \item \textbf{Deployment :}  Often many of these pieces are overlapped, for example, it’s common for development to be going on while the UI is being finalized, and it may even inform the UI design. Additionally, an application may be going into a stabilization phase at the same that new features are being added to a new version.

\end{itemize}

Furthermore, these phases can be used in any number of SDLC methodologies such as Agile, Spiral, Waterfall, etc.

Each of the these phases will be explained in more detail by the following sections.



\subsection{Inception}

The ubiquity and level of interaction people have with mobile devices means that nearly everyone has an idea for a mobile app. Mobile devices open up a whole new way to interact with computing, the web, and even corporate infrastructure.

\paragraph{}
The inception stage is all about defining and refining the idea for an app. In order to create a successful app, it’s important to ask some fundamental questions. Here are some things to consider before publishing an app in one of the public App Stores:

\begin{itemize}


 \item \textbf{Competitive Advantage:}Are there similar apps out there already? If so, how does this application differentiate from others?
For apps that will be distributed in an Enterprise:

\item \textbf{Infrastructure Integration:}What existing infrastructure will it integrate with or extend?


 Additionally, apps should be evaluated in the context of the mobile form factor:

 \item \textbf{Value:}  What value does this app bring users? How will they use it?
 \item \textbf{Form/Mobility:} How will this app work in a mobile form factor? How can I add value using mobile technologies such as location awareness, the camera, etc.?


\end{itemize}

To help with designing the functionality of an app, it can be useful to define Actors and Use Cases. Actors are roles within an application and are often users. Use cases are typically actions or intents.

For instance, a task tracking application might have two Actors: User and Friend. A User might Create a Task, and Share a Task with a Friend. In this case, creating a task and sharing a task are two distinct use cases that, in tandem with the Actors, will inform what screens you’ll need to build, as well as what business entities and logic will need to be developed.

Once an appropriate number of use cases and actors has been captured, it’s much easier to begin designing an application. Development can then focus on how to create the app, rather than what the app is or should do.

\subsection{Designing Mobile Applications}

Once the features and functionality of the app have been determined, the next step is start trying to solve the User Experience or UX.


UX Design

UX is usually done via wireframes or mockups using tools such as Balsamiq, Mockingbird, Visio, or just plain ol’ pen and paper. UX Mockups allow the UX to be designed without having to worry about the actual UI design:

When creating UX Mockups, it’s important to consider the Interface Guidelines for the various platforms that the app will target. The app should "feel at home" on each platform. The offical design guidelines for each platform are:

\begin{itemize}


 \item \textbf{Apple:} Human Interface Guidelines
 %https://developer.apple.com/ios/human-interface-guidelines/overview/design-principles/

 \item \textbf{Android:}  Design Guidelines
%https://developer.android.com/design/index.html

 \item \textbf{Windows Phone:}  Design library for Windows Phone

%https://msdn.microsoft.com/library/windows/apps/fa00461b-abe1-41d1-be87-0b0fe3d3389d(v=vs.105).aspx

\end{itemize}
For example, each app has a metaphor for switching between sections in an application. iOS uses a tabbar at the bottom of the screen, Android uses a tabbar at the top of the screen, and Windows Phone uses the Panorama view:

Additionally, the hardware itself also dictates UX decisions. For example, iOS devices have no physical back button, and therefore introduce the Navigation Controller metaphor:

%ref: https://developer.xamarin.com/guides/cross-platform/getting_started/introduction_to_mobile_sdlc/

\subsection{Development}
The development phase usually starts very early. In fact, once an idea has some maturation in the conceptual/inspiration phase, often a working prototype is developed that validates functionality, assumptions, and helps to give an understanding of the scope of the work.

In the rest of the tutorials, we’ll focus largely on the development phase.

% quick start

%----------------------------------------------------------------------------------------
%	Stabilization
%----------------------------------------------------------------------------------------
\subsection{Stabilization}

Stabilization is the process of working out the bugs in your app. Not just from a functional standpoint, e.g.: “It crashes when I click this button,” but also Usability and Performance. It’s best to start stabilization very early within the development process so that course corrections can occur before they become costly. Typically, applications go into Prototype, Alpha, Beta, and Release Candidate stages. Different people define these differently, but they generally follow the following pattern:

\begin{itemize}


 \item \textbf{Prototype:}   The app is still in proof-of-concept phase and only core functionality, or specific parts of the application are working. Major bugs are present.
\item \textbf{Alpha :} Core functionality is generally code-complete (built, but not fully tested). Major bugs are still present, outlying functionality may still not be present.
\item \textbf{Beta:}  Most functionality is now complete and has had at least light testing and bug fixing. Major known issues may still be present.
\item \textbf{Release Candidate :}All functionality is complete and tested. Barring new bugs, the app is a candidate for release to the wild.
\end{itemize}

It’s never too early to begin testing an application. For example, if a major issue is found in the prototype stage, the UX of the app can still be modified to accommodate it. If a performance issue is found in the alpha stage, it’s early enough to modify the architecture before a lot of code has been built on top of false assumptions.
Typically, as an application moves further along in the lifecycle, it’s opened to more people to try it out, test it, provide feedback, etc. For instance, prototype applications may only be shown or made available to key stakeholders, whereas release candidate applications may be distributed to customers that sign up for early access.
For early testing and deployment to relatively few devices, usually deploying straight from a development machine is sufficient. However, as the audience widens, this can quickly become cumbersome. As such, there are a number of test deployment options out there that make this process much easier by allowing you to invite people to a testing pool, release builds over the web, and provide tools that allow for user feedback.

Some of the most popular ones are:
\begin{itemize}

 \item \textbf{Testflight}  – This is an iOS product that allows you to distribute apps for testing as well as receive crash reports and usage information from your customers. This is included as part of iTunes connect, and is not available if you are part of an Apple Developer Enterprise membership.
 \item \textbf{LaunchPad (launchpadapp.com)}  – Designed for Android, this service is very similar to TestFlight.
 \item \textbf{Vessel (vessel.io)}  – A service for iOS and Android that lets you monitor usage, track customers and even do A/B testing from inside your app.
 \item \textbf{hockeyapp.com } - Provides a testing service for iOS, Android and Windows Phone.
\end{itemize}


%----------------------------------------------------------------------------------------
%	Distribution
%----------------------------------------------------------------------------------------
\subsection{Distribution}

\begin{itemize}
\item \textbf{Apple App Store}
Once the application has been stabilized, it’s time to get it out into the wild. There are a number of different distribution options, depending on the platform.

Xamarin.iOS and Objective-C apps are distributed in exactly the same way:

\begin{itemize}

\item \textbf{Apple App Store: } Apple’s App Store is a globally available online application repository that is built into Mac OS X via iTunes. It’s by far the most popular distribution method for applications and it allows developers to market and distribute their apps online with very little effort.
\item \textbf{In-House Deployment : } In-House deployment is meant for internal distribution of corporate applications that aren’t available publicly via the App Store.
\item \textbf{Ad-Hoc Deployment: }  Ad-hoc deployment is intended primarily for development and testing and allows you to deploy to a limited number of properly provisioned devices. When you deploy to a device via Xcode or Xamarin Studio, it is known as ad-hoc deployment.

\end{itemize}

\item \textbf{Android Google Play}

All Android applications must be signed before being distributed. Developers sign their applications by using their own certificate protected by a private key. This certificate can provide a chain of authenticity that ties an application developer to the applications that developer has built and released. It must be noted that while a development certificate for Android can be signed by a recognized certificate authority, most developers do not opt to utilize these services, and self-sign their certificates. The main purpose for certificates is to differentiate between different developers and applications. Android uses this information to assist with enforcement of delegation of permissions between applications and components running within the Android OS.

Unlike other popular mobile platforms, Android takes a very open approach to app distribution. Devices are not locked to a single, approved app store. Instead, anyone is free to create an app store, and most Android phones allow apps to be installed from these third party stores.

This allows developers a potentially larger yet more complex distribution channel for their applications. Google Play is Google’s official app store, but there are many others.

A few popular ones are: AppBrain , Amazon App Store for Android , Handango, jetJar

\item \textbf{Windows Store}

Windows Phone applications are distributed to users via the Windows Store. Developers submit their apps to the Windows Phone Dev Center for approval, after which they appear in the Store.

Microsoft provides detailed instructinos for deploying Windows Phone apps during development.

Follow these steps to publish apps for beta testing and release to the store. Developers can submit their apps and then provide an install link to testers, before the app is reviewed and published.

\end{itemize}
%----------------------------------------------------------------------------------------
%	Stabilization
%----------------------------------------------------------------------------------------

\section{Mobile Development Considerations}

While developing mobile applications isn’t fundamentally different that traditional web/desktop development in terms of process or architecture, there are some considerations to be aware of.

\subsection{Common Considerations}

\begin{itemize}

\item \textbf{Multitasking: }
There are two significant challenges to multitasking (having multiple applications running at once) on a mobile device. First, given the limited screen real estate, it is difficult to display multiple applications simultaneously. Therefore, on mobile devices only one app can be in the foreground at one time. Second, having multiple applications open and performing tasks can quickly use up battery power.

Each platform handles multitasking differently, which we’ll explore in a bit.


\item \textbf{Form Factor: }
Mobile devices generally fall into two categories, phones and tablets, with a few crossover devices in between. Developing for these form factors is generally very similar, however, designing applications for them can be very different. Phones have very limited screen space, and tablets, while bigger, are still mobile devices with less screen space than even most laptops. Because of this, mobile platform UI controls have been designed specifically to be effective on smaller form factors.


\item \textbf{ Device and OS Fragmentation: }
It’s important to take into account different devices throughout the entire software development lifecycle:

\begin{itemize}

\item \textbf{Conceptualization and Planning : } Keep in mind that hardware and features will vary from device to device, an application that relies on certain features may not work properly on certain devices. For example, not all devices have cameras, so if you’re building a video messaging application, some devices may be able to play videos, but not take them.
\item \textbf{Design: }  When designing an application’s User Experience (UX), pay attention to the different screen ratios and sizes across devices. Additionally, when designing an application’s User Interface (UI), different screen resolutions should be considered.
\item \textbf{Development : }When using a feature from code, the presence of that feature should always be tested first. For example, before using a device feature, such as a camera, always query the OS for the presence of that feature first. Then, when initializing the feature/device, make sure to request currently supported from the OS about that device and then use those configuration settings.
\item \textbf{Testing : } It’s incredibly important to test the application early and often on actual devices. Even devices with the same hardware specs can vary widely in their behavior.
\end{itemize}



\item \textbf{Limited Resources: }
Mobile devices get more and more powerful all the time, but they are still mobile devices that have limited capabilities in comparison to desktop or notebook computers. For instance, desktop developers generally don’t worry about memory capacities; they’re used to having both physical and virtual memory in copious quantities, whereas on mobile devices you can quickly consume all available memory just by loading a handful of high-quality pictures.

Additionally, processor-intensive applications such as games or text recognition can really tax the mobile CPU and adversely affect device performance.

Because of considerations like these, it’s important to code smartly and to deploy early and often to actual devices in order to validate responsiveness.
\end{itemize}

\subsection{iOS Considerations}

\begin{itemize}
\item \textbf{Multitasking: }
Multitasking is very tightly controlled in iOS, and there are a number of rules and behaviors that your application must conform to when another application comes to the foreground, otherwise your application will be terminated by iOS.


\item \textbf{Device-Specific Resources:}
Within a particular form factor, hardware can vary greatly between different models. For instance, some devices have a rear-facing camera, some also have a front-facing camera, and some have none.

Some older devices (iPhone 3G and older) don’t even allow multitasking.

Because of these differences between device models, it’s important to check for the presence of a feature before attempting to use it.


\item \textbf{OS Specific Constraints:}
In order to make sure that applications are responsive and secure, iOS enforces a number of rules that applications must abide by. In addition to the rules regarding multitasking, there are a number of event methods out of which your app must return in a certain amount of time, otherwise it will get terminated by iOS.

Also worth noting, apps run in what’s known as a Sandbox, an environment that enforces security constraints that restrict what your app can access. For instance, an app can read from and write to its own directory, but if it attempts to write to another app directory, it will be terminated.

\end{itemize}

\subsection{Android Considerations}

\begin{itemize}
\item \textbf{Multitasking: }
Multitasking in Android has two components; the first is the activity lifecycle. Each screen in an Android application is represented by an Activity, and there is a specific set of events that occur when an application is placed in the background or comes to the foreground. Applications must adhere to this lifecycle in order to create responsive, well-behaved applications. For more information, see the Activity Lifecycle guide.

The second component to multitasking in Android is the use of Services. Services are long-running processes that exist independent of an application and are used to execute processes while the application is in the background. For more information see the Creating Services guide.


\item \textbf{Many Devices \& Many Form Factors: }

Unlike iOS, which has a small set of devices, or even Windows Phone, which only runs on approved devices that meet a minimum set of platform requirements, Google doesn’t impose any limits on which devices can run the Android OS. This open paradigm results in a product environment populated by a myriad of different devices with very different hardware, screen resolutions and ratios, device features, and capabilities.

Because of the extreme fragmentation of Android devices, most people choose the most popular 5 or 6 devices to design and test for, and prioritize those.


\item \textbf{ Security Considerations:}
Applications in the Android OS all run under a distinct, isolated identity with limited permissions. By default, applications can do very little. For example, without special permissions, an application cannot send a text message, determine the phone state, or even access the Internet! In order to access these features, applications must specify in their application manifest file which permissions they would like, and when they’re being installed; the OS reads those permissions, notifies the user that the application is requesting those permissions, and then allows the user to continue or cancel the installation. This is an essential step in the Android distribution model, because of the open application store model, since applications are not curated the way they are for iOS, for instance. For a list of application permissions, see the Manifest Permissions reference article in the Android Documentation.

\end{itemize}


\subsection{Windows Phone Considerations}

\begin{itemize}
\item \textbf{Multitasking: }
Multitasking in Windows Phone also has two parts: the lifecycle for pages and applications, and background processes. Each screen in an application is an instance of a Page class, which has events associated with being made active or inactive (with special rules for handling the inactive state, or being “tombstoned”). For more information see the Execution Model Overview for Windows Phone documentation.

The second part is providing background agents for processing tasks even when the application is not running in the foreground. More information on scheduling periodic tasks or creating resource intensive background tasks can be found in the Background Agents Overview.


\item \textbf{DEVICE Capabilities: }

Although Windows Phone hardware is fairly homogeneous due to the strict guidelines provided by Microsoft, there are still components that are optional and therefore require special considering while coding. Optional hardware capabilities include the camera, compass and gyroscope. There is also a special class of low-memory (256MB) that requires special consideration, or developers can opt-out of low-memory support.


\item \textbf{Database: }

Both iOS and Android include the SQLite database engine that allows for sophisticated data storage that also works cross-platform. Windows Phone 7 did not include a database, while Windows Phone 7.1 and 8 include a local database engine that can only be queried with LINQ to SQL and does not support Transact-SQL queries. There is an open-source port of SQLite available that can be added to Windows Phone applications to provide familiar Transact-SQL support and cross-platform compatibility.


\item \textbf{Security Considerations: }

Windows Phone applications are run with a restricted set of permissions that isolates them from one another and limits the operations they can perform. Network access must be performed via specific APIs and inter-application communication can only be done via controlled mechanisms. Access to the file-system is also restricted; the Isolated Storage API provides key-value pair storage and the ability to create files and folders in a controlled fashion (refer to the Isolated Storage Overview for more information).

An application’s access to hardware and operating system features is controlled by the capabilities listed in its manifest file (similar to Android). The manifest must declare the features required by the application, so that users can see and agree to those permissions and also so that the operating system allows access to the APIs. Applications must request access to features like the contacts or appointments data, camera, location, media library and more. See Microsoft’s Application Manifest File documentation for additional information.

\end{itemize}

\subsection{Summary}

This guide gave an introduction to the SDLC as it relates to mobile development. It introduced general considerations for building mobile applications and examined a number of platform-specific considerations including design, testing, and deployment.
