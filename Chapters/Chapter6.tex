% !TEX root = ../main.tex
% Chapter 6

\chapter{ The Canteen Feedback system} % Main chapter title

\label{Chapter6} % For referencing the chapter elsewhere, use \ref{Chapter1}

%----------------------------------------------------------------------------------------

% Define some commands to keep the formatting separated from the content
% \newcommand{\keyword}[1]{\textbf{#1}}
% \newcommand{\tabhead}[1]{\textbf{#1}}
% \newcommand{\code}[1]{\texttt{#1}}
% \newcommand{\file}[1]{\texttt{\bfseries#1}}
% \newcommand{\option}[1]{\texttt{\itshape#1}}

%----------------------------------------------------------------------------------------
%----------------------------------------------------------------------------------------
%	Overview
%----------------------------------------------------------------------------------------

\section{Hypothetical Scenarios}


Building mobile applications can be as easy as opening up the IDE, throwing something together, doing a quick bit of testing, and submitting to an App Store – all done in an afternoon. Or it can be an extremely involved process that involves rigorous up-front design, usability testing, QA testing on thousands of devices, a full beta lifecycle, and then deployment a number of different ways.

In this document, we’re going to take a thorough introductory examination of building mobile applications, including:

Process – The process of software development is called the Software Development Lifecycle (SDLC). We’ll examine all phases of the SDLC with respect to mobile application development, including: Inspiration, Design, Development, Stabilization, Deployment, and Maintenance.
Considerations – There are a number of considerations when building mobile applications, especially in contrast to traditional web or desktop applications. We’ll examine these considerations and how they affect mobile development.
This document is intended to answer fundamental questions about mobile app development, for new and experienced application developers alike. It takes a fairly comprehensive approach to introducing most of the concepts you’ll run into during the entire Software Development Lifecycle (SDLC).



\section{ The Feedback system}

This section describes the actors in the betting system (see Figure 2-1), which has three groups of actors. The administrators, or bookmakers, administrates the system and create new bets. The football matches are users that simulate a football match. These users can start new bets and generate goals. The clients are user that places bets during a football match.


\section{ General requirements}

\section{Functional requirements}

\subsection{Administrator requirements}

The administrator shall be able to:

\begin{itemize}

\item \textbf{} log on to the system (3).

\item \textbf{}  add new users and administrate existing users (3).

\end{itemize}

\subsection{Client requirements}

The clients shall be able to:
\begin{itemize}
\item \textbf{} log on to the system (3).
\item \textbf{} be notified of new bets, place a bet and get feedback about the outcome of the bet (3).
\item \textbf{}  get the balance for his account (2

\end{itemize}

\section{Non-Functional Requirements}

\subsection{Administrator requirements}

\subsection{Client requirements}
